
    %%%%% %%%%% %%%%%
    %               %
    % New row       %
    %               %
    %%%%% %%%%% %%%%%
\begin{columns}[t]
  \begin{column}{0.44\textwidth}
    \begin{block}{\large Rationale and methodology}
%\small
\scriptsize
%\tiny
\textit{Phytophthora infestans} is the most devastating pathogen of potato.
We set out on a genomics project and realized that copy number variation (CNV) may be more important than single nucleotide polymorphisms.
Our project focused on:
\vspace{5mm}

\begin{itemize}
\item Determine copy number for all \textit{P. infestans} genes
\item Determine if certain categories of genes had more CNV
\item Ask if CNV occurred in other species of \textit{Phytophthora}
\end{itemize}

\vspace{10mm}

\tiny
We sequenced genomes of isolates of \textit{P. infestans} collected from Mexico and added these to genomes already published.
We used the method of Knaus and Gr\"unwald (2018) to assign copy number to genomic windows in \textit{P. infestans}.
We then used the genomic coordinates of each gene to determine which window and what its copy number was for all \textit{P. infestans} genes.
We also determined copy number for genomic windows of several other \textit{Phytophthora} species available at the SRA.
\vspace{20mm}
    \end{block}
  \end{column}

  \begin{column}{0.5\textwidth}
    \begin{block}{\large Copy number varies continuously}
      \begin{columns}
        \begin{column}{0.50\textwidth}
%\small
%\footnotesize
\scriptsize
%\tiny
\vspace{5mm}

Copy number determined for each gene did not result in isolates that were diploid or triploid.
Instead, we found a gradient of isolates from predominantly diploid to predominantly triploid.
This indicated that this was not simply a matter of ploidy, but instead was copy number variation within each genome.

\vspace{5mm}

Sexually reproducing isolates from Mexico were predominantly diploid while clonal lineages outside of Mexico tended to be have three copies of each gene suggesting a link between copy number and mode of reproduction.

%\vspace{1cm}
%          \begin{figure}
        \end{column}
        \begin{column}{0.47\textwidth}
          \includegraphics*[viewport=200 0 567 270, clip, height=20cm]{./figures/Fig3_V2.pdf}
        \end{column}
      \end{columns}
%          \caption{Allele balance is the frequency theat the most abundant and second most abundant allele were sequenced at.}
%          \end{figure}

    \end{block}

  \end{column}

\end{columns}


